% Options for packages loaded elsewhere
\PassOptionsToPackage{unicode}{hyperref}
\PassOptionsToPackage{hyphens}{url}
%
\documentclass[
  a4paper,
]{article}
\usepackage{amsmath,amssymb}
\usepackage{iftex}
\ifPDFTeX
  \usepackage[T1]{fontenc}
  \usepackage[utf8]{inputenc}
  \usepackage{textcomp} % provide euro and other symbols
\else % if luatex or xetex
  \usepackage{unicode-math} % this also loads fontspec
  \defaultfontfeatures{Scale=MatchLowercase}
  \defaultfontfeatures[\rmfamily]{Ligatures=TeX,Scale=1}
\fi
\usepackage{lmodern}
\ifPDFTeX\else
  % xetex/luatex font selection
\fi
% Use upquote if available, for straight quotes in verbatim environments
\IfFileExists{upquote.sty}{\usepackage{upquote}}{}
\IfFileExists{microtype.sty}{% use microtype if available
  \usepackage[]{microtype}
  \UseMicrotypeSet[protrusion]{basicmath} % disable protrusion for tt fonts
}{}
\makeatletter
\@ifundefined{KOMAClassName}{% if non-KOMA class
  \IfFileExists{parskip.sty}{%
    \usepackage{parskip}
  }{% else
    \setlength{\parindent}{0pt}
    \setlength{\parskip}{6pt plus 2pt minus 1pt}}
}{% if KOMA class
  \KOMAoptions{parskip=half}}
\makeatother
\usepackage{xcolor}
\usepackage[margin=1in]{geometry}
\usepackage{graphicx}
\makeatletter
\def\maxwidth{\ifdim\Gin@nat@width>\linewidth\linewidth\else\Gin@nat@width\fi}
\def\maxheight{\ifdim\Gin@nat@height>\textheight\textheight\else\Gin@nat@height\fi}
\makeatother
% Scale images if necessary, so that they will not overflow the page
% margins by default, and it is still possible to overwrite the defaults
% using explicit options in \includegraphics[width, height, ...]{}
\setkeys{Gin}{width=\maxwidth,height=\maxheight,keepaspectratio}
% Set default figure placement to htbp
\makeatletter
\def\fps@figure{htbp}
\makeatother
\setlength{\emergencystretch}{3em} % prevent overfull lines
\providecommand{\tightlist}{%
  \setlength{\itemsep}{0pt}\setlength{\parskip}{0pt}}
\setcounter{secnumdepth}{-\maxdimen} % remove section numbering
% definitions for citeproc citations
\NewDocumentCommand\citeproctext{}{}
\NewDocumentCommand\citeproc{mm}{%
  \begingroup\def\citeproctext{#2}\cite{#1}\endgroup}
\makeatletter
 % allow citations to break across lines
 \let\@cite@ofmt\@firstofone
 % avoid brackets around text for \cite:
 \def\@biblabel#1{}
 \def\@cite#1#2{{#1\if@tempswa , #2\fi}}
\makeatother
\newlength{\cslhangindent}
\setlength{\cslhangindent}{1.5em}
\newlength{\csllabelwidth}
\setlength{\csllabelwidth}{3em}
\newenvironment{CSLReferences}[2] % #1 hanging-indent, #2 entry-spacing
 {\begin{list}{}{%
  \setlength{\itemindent}{0pt}
  \setlength{\leftmargin}{0pt}
  \setlength{\parsep}{0pt}
  % turn on hanging indent if param 1 is 1
  \ifodd #1
   \setlength{\leftmargin}{\cslhangindent}
   \setlength{\itemindent}{-1\cslhangindent}
  \fi
  % set entry spacing
  \setlength{\itemsep}{#2\baselineskip}}}
 {\end{list}}
\usepackage{calc}
\newcommand{\CSLBlock}[1]{\hfill\break\parbox[t]{\linewidth}{\strut\ignorespaces#1\strut}}
\newcommand{\CSLLeftMargin}[1]{\parbox[t]{\csllabelwidth}{\strut#1\strut}}
\newcommand{\CSLRightInline}[1]{\parbox[t]{\linewidth - \csllabelwidth}{\strut#1\strut}}
\newcommand{\CSLIndent}[1]{\hspace{\cslhangindent}#1}
\usepackage{amsmath}
\usepackage{amsthm}
\usepackage{booktabs}
\usepackage{multirow}
\usepackage{placeins}
\newtheorem{definition}{Definition}
\DeclareMathOperator*{\argmax}{argmax}
\DeclareMathOperator*{\argmin}{argmin}
\DeclareMathOperator{\sgn}{sgn}
\ifLuaTeX
  \usepackage{selnolig}  % disable illegal ligatures
\fi
\usepackage{bookmark}
\IfFileExists{xurl.sty}{\usepackage{xurl}}{} % add URL line breaks if available
\urlstyle{same}
\hypersetup{
  pdftitle={Appendix},
  pdfauthor={Alex Zwanenburg, Steffen Löck},
  hidelinks,
  pdfcreator={LaTeX via pandoc}}

\title{Appendix}
\author{Alex Zwanenburg, Steffen Löck}
\date{2025-04-11}

\begin{document}
\maketitle

\section{Appendix A: Log-likelihood functions for location- and
scale-invariant power
transformation}\label{appendix-a-log-likelihood-functions-for-location--and-scale-invariant-power-transformation}

Location- and scale-invariant Box-Cox and Yeo-Johnson transformations
are parametrised using location \(x_0\) and scale \(s\) parameters, in
addition to transformation parameter \(\lambda\). This leads to the
following transformations. The location- and scale-invariant Box-Cox
transformation is:

\begin{equation}
\phi_{\text{BC}}^{\lambda, x_0, s} (x_i) = 
\begin{cases}
\left( \left(\frac{x_i - x_0}{s} \right)^\lambda - 1 \right) / \lambda & \text{if } \lambda \neq 0\\
\log\left[\frac{x_i - x_0}{s}\right] & \text{if } \lambda = 0
\end{cases}
\end{equation}

where \(x_i - x_0 > 0\). The location- and scale-invariant Yeo-Johnson
transformation is:

\begin{equation}
\phi_{\text{YJ}}^{\lambda, x_0, s} (x_i) = 
\begin{cases}
\left( \left( 1 + \frac{x_i - x_0}{s}\right)^\lambda - 1\right) / \lambda & \text{if } \lambda \neq 0 \text{ and } x_i - x_0 \geq 0\\
\log\left[1 + \frac{x_i - x_0}{s}\right] & \text{if } \lambda = 0 \text{ and } x_i - x_0 \geq 0\\
-\left( \left( 1 - \frac{x_i - x_0}{s}\right)^{2 - \lambda} - 1 \right) / \left(2 - \lambda \right) & \text{if } \lambda \neq 2 \text{ and } x_i - x_0 < 0\\
-\log\left[1 - \frac{x_i - x_0}{s}\right] & \text{if } \lambda = 2 \text{ and } x_i - x_0 < 0
\end{cases}
\end{equation}

The parameters of these power transformations can be optimised based by
maximising the log-likelihood function, under the assumption that the
transformed feature \(\phi^{\lambda, x_0, s} (\mathbf{X})\) follows a
normal distribution. The log-likelihood functions for conventional
Box-Cox and Yeo-Johnson transformations are well-known. However, the
introduction of scaling parameter \(s\) prevents their direct use. Here,
we first derive the general form of the log-likelihood functions, and
then derive their power-transformation specific definitions.

Let \(f(x_1, \ldots, x_n)\) be the probability density function of
feature \(\mathbf{X} = \{ x_1, \ldots, x_n\}\), and
\(f^{\lambda, x_0, s} (\phi^{\lambda, x_0, s}(x_1), \ldots, \phi^{\lambda, x_0, s}(x_n))\)
be the probability density function of the transformed feature
\(\phi^{\lambda, x_0, s} (\mathbf{X})\), that is assumed to follow a
normal distribution.

The two probability density functions are related as follows:

\begin{equation}
f^{\lambda, x_0, s}(x_1, \ldots, x_n) = f^{\lambda, x_0, s} (\phi^{\lambda, x_0, s}(x_1), \ldots, \phi^{\lambda, x_0, s}(x_n)) \left|\mathbf{J}\right|
\end{equation}

Where, \(\left|\mathbf{J}\right|\) is the determinant of Jacobian
\(\mathbf{J}\). The Jacobian takes the following form, with off-diagonal
elements \(0\):

\begin{equation}
\mathbf{J} =
\begin{bmatrix}
    \frac{\partial}{\partial x_1} \phi^{\lambda, x_0, s}(x_1) & 0 & \dots & 0 \\
    0 & \frac{\partial}{\partial x_2} \phi^{\lambda, x_0, s}(x_2) & \dots & 0 \\
    \vdots & \vdots  & \ddots &  \vdots \\
    0  & 0 & 0 & \frac{\partial}{\partial x_n} \phi^{\lambda, x_0, s}(x_n)
\end{bmatrix}
\end{equation}

Thus,
\(\left| \mathbf{J} \right| = \prod_{i=1}^n \frac{\partial}{\partial x_i} \phi^{\lambda, x_0, s}(x_i)\).

Since in our situation \(\{x_1, \ldots, x_n\}\) in
\(f^{\lambda, x_0, s}(x_1, \ldots, x_n)\) are considered fixed (i.e.,
known), \(f^{\lambda, x_0, s}(x_1, \ldots, x_n)\) may be considered a
likelihood function. The log-likelihood function
\(\mathcal{l}^{\lambda, x_0, s}\) is then:

\begin{equation}
\begin{split}
\mathcal{l}^{\lambda, x_0, s} & = \log f^{\lambda, x_0, s}(x_1, \ldots, x_n) \\
 & = \log \left[ f^{\lambda, x_0, s} (\phi^{\lambda, x_0, s}(x_1), \ldots, \phi^{\lambda, x_0, s}(x_n)) \right] + \log \left|\mathbf{J}\right| \\
 & = \log \left[ f^{\lambda, x_0, s} (\phi^{\lambda, x_0, s}(x_1), \ldots, \phi^{\lambda, x_0, s}(x_n)) \right] + \log \prod_{i=1}^n \frac{\partial}{\partial x_i} \phi^{\lambda, x_0, s}(x_i) \\
 & = -\frac{n}{2} \log \left[2 \pi \sigma^2 \right] -\frac{1}{2 \sigma^2} \sum_{i=1}^n \left( \phi^{\lambda, x_0, s}(x_i) - \mu \right)^2 + \sum_{i=1}^n \log \left[ \frac{\partial}{\partial x_i} \phi^{\lambda, x_0, s}(x_i)\right]
\end{split}
\end{equation}

With \(\mu\) the average of \(\phi^{\lambda, x_0, s}(\mathbf{X})\) and
\(\sigma^2\) its variance. The first two terms derive directly from the
log-likelihood function of a normal distribution, and are not specific
to the type of power transformation used. However, the final term
differs between Box-Cox and Yeo-Johnson transformations.

\subsection{Location- and scale-invariant Box-Cox
transformation}\label{location--and-scale-invariant-box-cox-transformation}

For the location- and scale-invariant Box-Cox transformation the partial
derivative is:

\begin{equation}
\begin{split}
\frac{\partial}{\partial x_i} \phi_{\text{BC}}^{\lambda, x_0, s}(x_i) & = \frac{1}{s} \left(\frac{x_i - x_0}{s} \right)^{\lambda-1} \\
 & = \frac{1} {s^\lambda} \left(x_i - x_0 \right)^{\lambda - 1}
\end{split}
\end{equation}

Thus the final term in \(\mathcal{l}_{\text{BC}}^{\lambda, x_0, s}\) is:

\begin{equation}
\begin{split}
\sum_{i=1}^n \log \frac{\partial}{\partial x_i} \phi_{\text{BC}}^{\lambda, x_0, s}(x_i) & = \sum_{i=1}^n \log \left[ s^{-\lambda} (x_i - x_0)^{\lambda - 1} \right] \\
& = \sum_{i=1}^n \log \left[s^{-\lambda} \right] + \log \left[ (x_i - x_0)^{\lambda - 1} \right]\\
& = -n \lambda \log s + \left( \lambda - 1 \right) \sum_{i=1}^n \log \left[ x_i - x_0 \right]
\end{split}
\end{equation}

This leads to the following log-likelihood:

\begin{equation}
\begin{split}
\mathcal{l}_{\text{BC}}^{\lambda, x_0, s} = & -\frac{n}{2} \log \left[2 \pi \sigma^2 \right] -\frac{1}{2 \sigma^2} \sum_{i=1}^n \left( \phi^{\lambda, x_0, s}(x_i) - \mu \right)^2 \\
& -n \lambda \log s + \left( \lambda - 1 \right) \sum_{i=1}^n \log \left[ x_i - x_0 \right]
\end{split}
\end{equation}

Similarly to Raymaekers and Rousseeuw (2024), sample weights \(w_i\) are
introduced to facilitate robust power transformations. The weighted
log-likelihood of the location- and scale-invariant Box-Cox
transformation is:

\begin{equation}
\begin{split}
\mathcal{l}_{\text{rBC}}^{\lambda, x_0, s} = & -\frac{1}{2} \left(\sum_{i=1}^n w_i \right) \log \left[ 2 \pi \sigma_w^2 \right] -\frac{1}{2 \sigma_w^2} \sum_{i=1}^n w_i \left( \phi^{\lambda, x_0, s}(x_i) - \mu_w \right)^2 \\
& - \lambda \left( \sum_{i=1}^n w_i \right) \log s + \left( \lambda - 1 \right) \sum_{i=1}^n w_i \log \left[ x_i - x_0 \right]
\end{split}
\end{equation}

where \(\mu_w\) and \(\sigma^2_w\) are the weighted mean and weighted
variance of the Box-Cox transformed feature
\(\phi_{\text{BC}}^{\lambda, x_0, s} (\mathbf{X})\), respectively:

\begin{equation}
\sigma_w^2 = \frac{\sum_{i=1}^n w_i \left(\phi_{\text{BC}}^{\lambda, x_0, s} (x_i) - \mu_w \right)^2}{\sum_{i=1}^n w_i} \quad \text{with } \mu_w = \frac{\sum_{i=1}^n \phi_{\text{BC}}^{\lambda, x_0, s} (x_i)} {\sum_{i=1}^n w_i}
\end{equation}

\subsection{Location- and scale-invariant Yeo-Johnson
transformation}\label{location--and-scale-invariant-yeo-johnson-transformation}

For the location- and scale-invariant Yeo-Johnson transformation, the
partial derivative is:

\begin{equation}
\frac{\partial}{\partial x_i} \phi_{\text{YJ}}^{\lambda, x_0, s}(x_i) =
\begin{cases}
\frac{1}{s} \left(1 + \frac{x_i - x_0}{s}\right)^{\lambda - 1} & \text{if } x_i - x_0 \geq 0\\
\frac{1}{s} \left(1 - \frac{x_i - x_0}{s}\right)^{1 - \lambda} & \text{if } x_i - x_0 < 0
\end{cases}
\end{equation}

Thus the final term in \(\mathcal{l}_{\text{YJ}}^{\lambda, x_0, s}\) is:

\begin{equation}
\begin{split}
\sum_{i=1}^n \log \frac{\partial}{\partial x_i} \phi_{\text{YJ}}^{\lambda, x_0, s}(x_i) & = - n \log s + (\lambda - 1) \sum_{i=1}^n \sgn(x_i - x_0) \log \left[1 + \frac{|x_i - x_0|}{s} \right]
\end{split}
\end{equation}

This leads to the following log-likelihood:

\begin{equation}
\begin{split}
\mathcal{l}_{\text{YJ}}^{\lambda, x_0, s} = & -\frac{n}{2} \log\left[2 \pi \sigma^2\right] -\frac{1}{2 \sigma^2} \sum_{i=1}^n \left( \phi^{\lambda, x_0, s}(x_i) - \mu \right)^2 \\
& - n \log s + (\lambda - 1) \sum_{i=1}^n \sgn(x_i - x_0) \log \left[1 + \frac{|x_i - x_0|}{s} \right]
\end{split}
\end{equation}

The weighted log-likelihood for location- and scale-invariant
Yeo-Johnson transformation is:

\begin{equation}
\begin{split}
\mathcal{l}_{\text{rYJ}}^{\lambda, x_0, s} = & -\frac{1}{2} \left(\sum_{i=1}^n w_i \right) \log \left[ 2 \pi \sigma_w^2 \right] -\frac{1}{2 \sigma_w^2} \sum_{i=1}^n w_i \left( \phi^{\lambda, x_0, s}(x_i) - \mu_w \right)^2 \\
& - \left( \sum_{i=1}^n w_i \right) \log s + (\lambda - 1) \sum_{i=1}^n w_i \sgn(x_i - x_0) \log \left[1 + \frac{|x_i - x_0|}{s} \right]
\end{split}
\end{equation}

where \(\mu_w\) and \(\sigma^2_w\) are the weighted mean and weighted
variance of the Yeo-Johnson transformed feature
\(\phi_{\text{YJ}}^{\lambda, x_0, s} (\mathbf{X})\):

\begin{equation}
\sigma_w^2 = \frac{\sum_{i=1}^n w_i \left(\phi_{\text{YJ}}^{\lambda, x_0, s} (x_i) - \mu_w \right)^2}{\sum_{i=1}^n w_i} \quad \text{with } \mu_w = \frac{\sum_{i=1}^n \phi_{\text{YJ}}^{\lambda, x_0, s} (x_i)} {\sum_{i=1}^n w_i}
\end{equation}

\section{Appendix B: Optimisation of transformation
parameters}\label{appendix-b-optimisation-of-transformation-parameters}

Maximum likelihood estimation (MLE) is commonly used to optimise
parameters for power transformation. Generally, optimisation requires
minimisation or maximisation of a criterion. In MLE, the maximised
criterion is the log-likelihood function of the normal distribution.
Here, we investigate power transformation using optimisation criteria
that are closely related to test statistics for normality tests.

Let \(\mathbf{X}\) be a feature with ordered feature values, and
\(\mathbf{Y}^\lambda =\phi^{\lambda} \left(\mathbf{X} \right)\) and
\(\mathbf{Y}^{\lambda, x_0, s} =\phi^{\lambda, x_0, s} \left(\mathbf{X} \right)\)
its transformed values using conventional and shift and scale invariant
power transformations, respectively. Since power transformations are
monotonic, \(\mathbf{Y}\) will likewise be ordered.

Below we will focus on criteria based on the empirical density function
and those based on skewness and kurtosis of the transformed featured.
Other potential criteria, such as the Shapiro-Wilk test statistic
(Shapiro and Wilk 1965) are not investigated here. In the case of the
Shapiro-Wilk test statistic this is because of lack of scalability to
features with many (\(> 5000\)) instances, and because adapting the test
statistic to include weights is not straightforward.

\subsection{Empirical density function-based
criteria}\label{empirical-density-function-based-criteria}

The first class of criteria is based on the empirical distribution
function (EDF). Transformation parameters are then fit through
minimisation of the distance between the empirical distribution function
\(F_{\epsilon}\) and the cumulative density function (CDF) of the normal
distribution \(F_{\mathcal{N}}\). Let
\(F_{\epsilon}\left(x_i \right) = \frac{i - 1/3}{n + 1/3}\) be the
empirical probability of instance \(i\). The normal distribution is
parametrised by location parameter \(\mu\) and scale parameter
\(\sigma\), both of which have to be estimated from the data. For
non-robust power transformations, \(\mu\) and \(\sigma\) are sample mean
and sample standard deviation, respectively. For robust power
transformations, we estimate \(\mu\) and \(\sigma\) as Huber M-estimates
of location and scale of the transformed feature
\(\phi^{\lambda, x_0, s} (\mathbf{X})\) (Huber 1981).

\subsubsection{Anderson-Darling
criterion}\label{anderson-darling-criterion}

The Anderson-Darling criterion is based on the empirical distribution
function of \(\mathbf{X}\). We define this criterion as follows:

\begin{equation}
U_{\text{AD}} \left(\mathbf{X}, \lambda, x_0 \right) = \frac{1}{\sum_{i=1}^n w_i} \sum_{i=1}^n w_i \frac{\left( F_{\epsilon}\left(x_i \right) - F_{\mathcal{N}} \left(\phi^{\lambda, x_0, s} \left(x_i \right); \mu, \sigma \right) \right)^2} {F_{\mathcal{N}} \left(\phi^{\lambda, x_0, s} \left(x_i \right); \mu, \sigma \right) \left(1 - F_{\mathcal{N}} \left(\phi^{\lambda, x_0, s} \left(x_i \right); \mu, \sigma \right) \right) }
\end{equation}

Here \(w_i\) are weights, and \(\mu\) and \(\sigma\) are location and
scale parameters. For non-robust power transformations, all \(w_i = 1\).
Note that this criterion is not the same as the Anderson-Darling test
statistic (Anderson and Darling 1952), which involves solving (or
approximating) an integral function, contains an extra scalar
multiplication term, and does not include weights. The Anderson-Darling
criterion seeks to minimise the squared Euclidean distance between the
EDF and the normal CDF, with differences at the upper and lower end of
the normal CDF receiving more weight than those at the the centre of the
CDF.

\subsubsection{Cramér-von Mises
criterion}\label{cramuxe9r-von-mises-criterion}

The Cramér-von Mises criterion is also based on the empirical
distribution function of \(\mathbf{X}\). We define the Cramér-von Mises
criterion as follows:

\begin{equation}
U_{\text{CvM}} \left(\mathbf{X}, \lambda, x_0 \right) = \frac{1}{\sum_{i=1}^n w_i} \sum_{i=1}^n w_i \left( F_{\epsilon}\left(x_i \right) - F_{\mathcal{N}} \left(\phi^{\lambda, x_0, s} \left(x_i \right); \mu, \sigma \right) \right)^2
\end{equation}

Here \(w_i\) are weights, and \(\mu\) and \(\sigma\) are location and
scale parameters. For non-robust power transformations, all \(w_i = 1\).
The criterion is similar to the Cramér-von Mises test statistic Mises
(1928), aside from a additive scalar value and the introduction of
weights. This criterion, like the Anderson-Darling criterion, seeks to
minimise the squared Euclidean distance between the EDF and the normal
CDF. Unlike the Anderson-Darling criterion, this criterion weights all
instances equally.

For conventional power transformations with a fixed shift parameter, the
transformation \(\phi^{\lambda, x_0, s} (\mathbf{X})\) may be
substituted by \(\phi^{\lambda} (\mathbf{X})\) in the definition of the
Cramér-von Mises criterion.

\subsection{Skewness-kurtosis-based
criteria}\label{skewness-kurtosis-based-criteria}

The second class of criteria seeks to reduce skewness and (excess)
kurtosis of the transformed feature \(\mathbf{Y}\). We will first define
the location \(\mu\) and scale \(\sigma\) of the the transformed as
these are required for computing skewness and kurtosis. Here, \(\mu\) is
defined as:

\begin{equation}
\mu = \frac{\sum_{i=1}^n \phi^{\lambda, x_0, s} \left(x_i \right)} {\sum_{i=1}^n w_i}
\end{equation}

The location, or mean, is weighted using weights \(w_i\). For non-robust
transformations, \(w_i = 1\). Then, \(\sigma^2\) is defined as:

\begin{equation}
\sigma^2 = \frac{\sum_{i=1}^n w_i \left(\phi^{\lambda, x_0, s} \left( x_i \right) - \mu \right)^2}{\sum_{i=1}^n w_i}
\end{equation}

Skewness is defined as:

\begin{equation}
s = \frac{\sum_{i=1}^n w_i \left(\phi^{\lambda, x_0, s} \left( x_i \right) - \mu \right)^3}{\sigma^3 \sum_{i=1}^n w_i}
\end{equation}

Kurtosis is defined as:

\begin{equation}
k = \frac{\sum_{i=1}^n w_i \left(\phi^{\lambda, x_0, s} \left( x_i \right) - \mu \right)^4}{\sigma^4 \sum_{i=1}^n w_i}
\end{equation}

\subsubsection{D'Agostino criterion}\label{dagostino-criterion}

The D'Agostino criterion defined here follows the D'Agostino \(K^2\)
test statistic (D'Agostino and Belanger 1990). This test statistic is
composed of two separate test statistics, one of which is related to
skewness, and the other to kurtosis. Both test statistics are computed
in several steps. Let us first define \(\nu=\sum_{i=1}^n w_i\). Thus for
non-robust power transformations, \(\nu = n\).

For the skewness test statistic we first compute (D'Agostino and
Belanger 1990):

\begin{equation}
\beta_1 = s \sqrt{ \frac{\left(\nu + 1\right) \left(\nu + 3\right)} {6 \left(\nu - 2\right)} }
\end{equation}

\begin{equation}
\beta_2 = 3 \frac{\left(\nu^2 + 27\nu - 70\right) \left(\nu + 1\right) \left(\nu + 3\right)} {\left(\nu - 2\right) \left(\nu + 5\right) \left(\nu + 7\right) \left(\nu + 9\right)}
\end{equation}

\begin{equation}
\alpha = \sqrt{\frac{2} {\sqrt{2 \beta_2 - 2} - 2}}
\end{equation}

\begin{equation}
\delta = \frac{1}{\sqrt{\log \left[\sqrt{-1 + \sqrt{2 * \beta_2 - 2}} \right]}}
\end{equation}

The skewness test statistic is then:

\begin{equation}
Z_s = \delta \log\left[\frac{\beta_1}{\alpha} + \sqrt{\frac{\beta_1^2}{\alpha^2} + 1} \right]
\end{equation}

For the kurtosis test statistic we first compute (D'Agostino and
Belanger 1990; Anscombe and Glynn 1983):

\begin{equation}
\beta_1 = 3 \frac{\nu - 1}{\nu + 1}
\end{equation}

\begin{equation}
\beta_2 = 24 \nu \frac{\left(\nu - 2\right)\left(\nu - 3\right)}{\left(\nu + 1\right)^2 \left(\nu + 3\right) \left(\nu + 5\right)}
\end{equation}

\begin{equation}
\beta_3 = 6 \frac{\nu^2 - 5 \nu + 2}{\left(\nu + 7\right) \left(\nu + 9\right)} \sqrt{6 \frac{\left(\nu + 3\right) \left(\nu + 5\right)}{\nu \left(\nu - 2\right) \left(\nu - 3 \right)}}
\end{equation}

\begin{equation}
\alpha_1 = 6 + \frac{8}{\beta_3} \left[\frac{2}{\beta_3} + \sqrt{1 + \frac{4}{\beta_3^2}} \right]
\end{equation}

\begin{equation}
\alpha_2 = \frac{k - \beta_1}{\sqrt{\beta_2}}
\end{equation}

The kurtosis test statistic is then:

\begin{equation}
Z_k = \sqrt{\frac{9 \alpha_1}{2}} \left[ 1 - \frac{2}{9 \alpha_1} - \left(\frac{1 - 2 / \alpha_1}{1 + \alpha_2 \sqrt{2 / \left(\alpha_1 - 4 \right)}} \right)^{1 / 3}  \right]
\end{equation}

The D'Agostino \(K^2\) test statistic and our criterion are the same,
and are defined as:

\begin{equation}
U_{\text{DA}} \left(\mathbf{X}, \lambda, x_0 \right) = Z_s^2 + Z_k^2
\end{equation}

The main difference between the test statistic as originally formulated,
and the criterion proposed here is the presence of weights for robust
power transformation.

\subsubsection{Jarque-Bera criterion}\label{jarque-bera-criterion}

The second criterion based on skewness and kurtosis is the Jarque-Bera
criterion. It is relatively simple to compute compared to the D'Agostino
criterion:

\begin{equation}
U_{\text{JB}} \left(\mathbf{X}, \lambda, x_0 \right) = s^2 + \left(k - 3\right)^2 / 4
\end{equation}

The main difference between the above criterion and the Jarque-Bera test
statistic (Jarque and Bera 1980) is that a scalar multiplication is
absent.

\subsection{Optimisation using non-MLE
criteria}\label{optimisation-using-non-mle-criteria}

Each of the above criteria can be used for optimisation, i.e.:

\begin{equation}
\left\{ \hat{\lambda}, \hat{x}_0, \hat{s}_0 \right\} = \argmin_{\lambda, x_0, s} U\left(\mathbf{X}, \lambda, x_0, s \right)
\end{equation}

For conventional power transformations with fixed location and scale
parameters, the transformation \(\phi^{\lambda, x_0, s} (\mathbf{X})\)
may be substituted by \(\phi^{\lambda} (\mathbf{X})\), or equivalently,
\(x_0\) and \(s\) may be fixed:

\begin{equation}
\left\{ \hat{\lambda}\right\} = \argmin_{\lambda} U\left(\mathbf{X}, \lambda; x_0, s \right)
\end{equation}

\subsection{Simulations with other optimisation
criteria}\label{simulations-with-other-optimisation-criteria}

Invariance of location- and scale-invariant power transformations was
assessed using the optimisation criteria in
\href{Appendix\%20B:\%20Optimisation\%20of\%20transformation\%20parameters}{Appendix
B}. This follows the simulation in the main manuscript, where MLE was
used for optimization. In short, we first randomly drew \(10000\) values
from a normal distribution:
\(\mathbf{X}_{\text{normal}} = \left\{x_1, x_2, \ldots, x_{10000} \right\} \sim \mathcal{N}\left(0, 1\right)\),
or equivalently
\(\mathbf{X}_{\text{normal}} = \left\{x_1, x_2, \ldots, x_{10000} \right\} \sim \mathcal{AGN}\left(0, 1/\sqrt{2}, 0.5, 2\right)\).
The second distribution was a right-skewed normal distribution
\(\mathbf{X}_{\text{right}} = \left\{x_1, x_2, \ldots, x_{10000} \right\} \sim \mathcal{AGN}\left(0, 1/\sqrt{2}, 0.2, 2\right)\).
The third distribution was a left-skewed normal distribution
\(\mathbf{X}_{\text{left}} = \left\{x_1, x_2, \ldots, x_{10000} \right\} \sim \mathcal{AGN}\left(0, 1/\sqrt{2}, 0.8, 2\right)\).

We then computed transformation parameter \(\lambda\) using the original
definitions (equations \ref{eqn:box-cox-original} and
\ref{eqn:yeo-johnson-original}) and the location- and scale-invariant
definitions (equations \ref{eqn:box-cox-invariant} and
\ref{eqn:yeo-johnson-invariant}) for each distribution using different
optimisation criteria. To assess location invariance, a positive value
\(d_{\text{shift}}\) was added to each distribution with
\(d_{\text{shift}} \in [1, 10^6]\). Similarly, to assess scale
invariance, each distribution was multiplied by a positive value
\(d_{\text{scale}}\), where \(d_{\text{scale}} \in [1, 10^6]\).

The results are shown in Figure
\ref{fig:shifted-distributions-appendix}.

\begin{figure}

{\centering \includegraphics{manuscript_appendix_files/figure-latex/shifted-distributions-appendix-1} 

}

\caption{Invariant power transformation produces transformation parameters that are invariant to location and scale. Samples were drawn from normal, right-skewed and left-skewed distributions, respectively, which then underwent a shift $d_{\text{shift}}$ or multiplication by $d_{\text{scale}}$. Estimates of the transformation parameter $\lambda$ for the conventional power transformations show strong dependency on the overall location and scale of the distribution and the optimisation criterion, whereas estimates obtained for the location- and scale-invariant power transformations are constant. For location- and scale-invariant power transformations, the Anderson-Darling criterion leads to unstable estimates of $\lambda$ for skewed distributions, possibly due to large weights being assigned to samples at the upper and lower ends of the distribution.}\label{fig:shifted-distributions-appendix}
\end{figure}

\section{Appendix D: Central normality test and empirical central
normality test critical
values}\label{appendix-d-central-normality-test-and-empirical-central-normality-test-critical-values}

Critical values for the central normality test and its empirical variant
are found in Table \ref{tab:central-normality-critical-statistic} and
Table \ref{tab:empirical-central-normality-critical-statistic},
respectively.

\begin{table}
\begin{center}
\caption{Critical values of test statistic $\tau_{\alpha, n, \kappa = 0.80}$ for
\textbf{central normality} at $\kappa = 0.80$, as a function of significance 
level $\alpha$ and number of instances $n$. The values shown must be divided by $100$.}
\label{tab:central-normality-critical-statistic}
\begin{tabular}{l | r r r r r r r r r}

\toprule
$n$ \textbackslash $\alpha$ & 0.001 & 0.01 & 0.025 & 0.05 & 0.1 & 0.2 & 0.5 & 0.8 & 0.9 \\

\midrule
    5 & 210.79 & 95.75 & 66.27 & 49.69 & 35.93 & 27.60 & 19.29 & 13.41 & 11.00 \\
   10 &  48.04 & 31.44 & 26.96 & 23.91 & 21.11 & 18.16 & 13.66 & 10.14 &  8.67 \\
   20 &  29.13 & 22.13 & 19.75 & 17.94 & 16.00 & 13.81 & 10.54 &  8.18 &  7.18 \\
   50 &  17.31 & 14.01 & 12.63 & 11.48 & 10.28 &  9.00 &  7.00 &  5.54 &  4.94 \\
  100 &  12.19 & 10.04 &  9.02 &  8.21 &  7.38 &  6.47 &  5.04 &  3.99 &  3.56 \\
  200 &   8.68 &  7.11 &  6.40 &  5.83 &  5.22 &  4.57 &  3.59 &  2.86 &  2.56 \\ 
  500 &   5.61 &  4.54 &  4.08 &  3.70 &  3.33 &  2.91 &  2.28 &  1.82 &  1.63 \\
 1000 &   3.93 &  3.19 &  2.87 &  2.60 &  2.33 &  2.05 &  1.61 &  1.29 &  1.15 \\
 2000 &   2.71 &  2.24 &  2.02 &  1.84 &  1.66 &  1.46 &  1.14 &  0.91 &  0.82 \\
 5000 &   1.73 &  1.42 &  1.28 &  1.17 &  1.05 &  0.92 &  0.72 &  0.58 &  0.52 \\
10000 &   1.22 &  1.00 &  0.90 &  0.82 &  0.74 &  0.65 &  0.51 &  0.41 &  0.37 \\
\bottomrule
\end{tabular}
\end{center}
\end{table}

\begin{table}
\begin{center}
\caption{Critical values of test statistic $\tau_{\alpha, n, \kappa = 0.70}$ for
\textbf{empirical central normality} at $\kappa = 0.70$, as a function of significance 
level $\alpha$ and number of instances $n$. The values shown must be divided by $100$.}
\label{tab:empirical-central-normality-critical-statistic}
\begin{tabular}{l | r r r r r r r r r}

\toprule
$n$ \textbackslash $\alpha$ & 0.001 & 0.01 & 0.025 & 0.05 & 0.1 & 0.2 & 0.5 & 0.8 & 0.9 \\

\midrule
    5 & 159.27 & 66.78 & 57.06 & 50.55 & 43.42 & 34.89 & 22.28 & 14.92 & 12.10 \\
   10 &  41.82 & 30.82 & 27.01 & 24.31 & 21.58 & 18.53 & 13.82 & 10.23 &  8.75 \\
   20 &  29.44 & 23.65 & 21.23 & 19.16 & 17.01 & 14.67 & 11.10 &  8.42 &  7.35 \\
   50 &  18.95 & 15.56 & 13.96 & 12.74 & 11.35 &  9.82 &  7.51 &  5.83 &  5.16 \\ 
  100 &  13.49 & 11.00 &  9.91 &  8.98 &  8.03 &  7.04 &  5.43 &  4.25 &  3.76 \\
  200 &   9.65 &  7.99 &  7.22 &  6.64 &  5.96 &  5.21 &  4.05 &  3.17 &  2.81 \\
  500 &   6.37 &  5.32 &  4.82 &  4.44 &  4.03 &  3.57 &  2.79 &  2.18 &  1.92 \\
 1000 &   4.64 &  3.95 &  3.65 &  3.39 &  3.11 &  2.78 &  2.20 &  1.73 &  1.53 \\
 2000 &   3.61 &  3.18 &  2.96 &  2.78 &  2.57 &  2.32 &  1.89 &  1.50 &  1.33 \\
 5000 &   2.78 &  2.48 &  2.36 &  2.24 &  2.11 &  1.95 &  1.66 &  1.39 &  1.25 \\
10000 &   2.39 &  2.18 &  2.08 &  2.00 &  1.91 &  1.81 &  1.59 &  1.39 &  1.28 \\
\bottomrule
\end{tabular}
\end{center}
\end{table}

\begin{figure}

{\centering \includegraphics{manuscript_appendix_files/figure-latex/empirical-central-normality-critical-statistics-1} 

}

\caption{Critical test statistic $\tau_{\alpha, n, \kappa = 0.80}$ of the (empirical) central normality test as function of $n$ for several values of significance level $\alpha$. The critical test statistics for central normality test are determined using fully normal data, whereas the statistics for the empirical variant are determined using centrally normal data, i.e. with fully normal data where 10\% of elements are replaced by outliers. emp: empirical}\label{fig:empirical-central-normality-critical-statistics}
\end{figure}

\section{Appendix E: Assessing transformations using simulated
data}\label{appendix-e-assessing-transformations-using-simulated-data}

\subsection{Ranking Box-Cox
transformations}\label{ranking-box-cox-transformations}

Box-Cox transformations were assessed in the same manner as Yeo-Johnson
transformations. The results are shown in Table
\ref{tab:comparison_methods_simulations_box_cox}.

\begin{table}
\begin{center}
\caption{
Comparison of average rank between Box-Cox transformation methods based on either residual error (without outliers) or residual error of the central portion
(with outliers; $\kappa = 0.80$) over 3 datasets with 10000 sequences each. The clean dataset consists of sequences derived through inverse Box-Cox transformation
of data sampled from a standard normal distribution. The dirty dataset contains sequences sampled from asymmetric generalised normal distributions.
The shifted dataset also contains sequences sampled from asymmetric generalised normal distributions, but centred at 100, and scaled by 0.001.
If necessary, each sequence was shifted so that every element had a strictly positive value. 
Several transformation methods include normalisation before transformation, indicated by z-score normalisation (norm.) or robust scaling.
A rank of 1 is the best and a rank of 9 the worst. For each dataset, the best ranking transformation is marked in bold.
}
\label{tab:comparison_methods_simulations_box_cox}
\begin{tabular}{l | l r r r r r r}

\toprule
& dataset: & \multicolumn{2}{c}{clean} & \multicolumn{2}{c}{dirty} & \multicolumn{2}{c}{shifted} \\
transformation & outliers: & no & yes & no & yes & no & yes \\

\midrule

none                                  & &         7.65  &         6.96  &         7.28  &         6.58  &         7.45  &         6.78 \\
conventional                          & &         3.10  &         5.98  &         5.42  &         6.16  &         6.49  &         6.10 \\
conventional (z-score norm.)          & &         4.19  &         6.28  &         4.85  &         6.33  &         4.38  &         5.85 \\
conventional (robust scaling)         & &         4.20  &         6.28  &         4.78  &         6.33  &         4.30  &         5.86 \\
Raymaekers-Rousseeuw                  & &         4.46  &         3.55  &         5.39  &         3.69  &         6.29  &         5.84 \\ 
Raymaekers-Rousseeuw (z-score norm.)  & &         6.21  &         3.83  &         4.95  &         3.85  &         4.51  &         3.50 \\ 
Raymaekers-Rousseeuw (robust scaling) & &         6.12  &         3.82  &         4.97  &         3.85  &         4.52  &         3.50 \\
invariant                             & & \textbf{2.99} &         5.55  & \textbf{3.65} &         6.01  & \textbf{3.52} &         5.57 \\
robust invariant                      & &         6.09  & \textbf{2.75} &         3.71  & \textbf{2.20} &         3.55  & \textbf{2.00} \\

\bottomrule
\end{tabular}
\end{center}
\end{table}

\subsection{Examples using clean data}\label{examples-using-clean-data}

Robust transformations are hypothesised to have a cost in efficiency for
data without outliers, i.e.~clean data. Here we draw nine sequences with
elements randomly drawn from a standard normal distribution
\(\mathcal{N}(0,1)\). Subsequently, we perform an inverse transformation
\(\left(\phi^{\lambda, 0, 1}\right)^{-1}\), with
\(\lambda \in \left(0, 2\right)\).

The sequences drawn resulted from the permutations of the number of
elements of each sequence (\(n \in \{30, 100, 500\}\)) and
transformation parameter for the inverse transformation
\(\lambda \in \{0.1, 1.0, 1.9\}\). Each sequence then underwent power
transformation using Box-Cox and Yeo-Johnson transformations. For
Box-Cox transformations, each sequence was shifted prior to inverse
transformation so that every element after inverse transformation would
be strictly positive.

Results for Box-Cox transformations are shown in Table
\ref{tab:clean-transformation-appendix-residuals-bc},
\ref{tab:clean-transformation-appendix-lambda-bc}, and
\ref{tab:clean-transformation-appendix-p-value-bc}. Additionally,
results for Yeo-Johnson transformations are shown in Table
\ref{tab:clean-transformation-appendix-residuals-yj},
\ref{tab:clean-transformation-appendix-lambda-yj}, and
\ref{tab:clean-transformation-appendix-p-value-yj}. As may be observed,
in these examples robust location- and shift-invariant transformations
have higher residual errors because of low weights being assigned to
tails of a distribution. This is also shown in Tables
\ref{tab:comparison_methods_simulations_yeo_johnson} and
\ref{tab:comparison_methods_simulations_box_cox} for clean data without
outliers.

\begin{table}
\begin{center}
\caption{Residual errors for features from simulated clean data without outliers after Yeo-Johnson transformation to normality.
Several transformation methods include normalisation before transformation, indicated by z-score normalisation (norm.) or robust scaling.}
\label{tab:clean-transformation-appendix-residuals-yj}
\small{
\begin{tabular}{l | l r r r r r r r r r}

\toprule
& n: & \multicolumn{3}{c}{30} & \multicolumn{3}{c}{100} & \multicolumn{3}{c}{500} \\
transformation & $\lambda$: & 0.1 & 1.0 & 1.9 & 0.1 & 1.0 & 1.9 & 0.1 & 1.0 & 1.9 \\

\midrule

none                                  & & 4.0 & 3.4 & 4.6 & 40.1 &  8.9 & 20.8 & 184.6 & 10.1 & 191.8 \\
conventional                          & & 2.7 & 3.3 & 3.6 &  6.5 &  8.1 &  5.6 &  14.2 & 10.1 &  16.1 \\
conventional (z-score norm.)          & & 2.5 & 3.3 & 3.7 &  9.8 &  8.1 &  5.6 &  24.1 & 10.1 &  15.9 \\
conventional (robust scaling)         & & 2.6 & 3.3 & 3.7 &  7.4 &  8.2 &  5.6 &  15.5 & 10.1 &  14.0 \\
Raymaekers-Rousseeuw                  & & 3.8 & 3.3 & 3.6 &  6.5 &  8.5 &  5.6 &  14.4 & 11.0 &  17.6 \\
Raymaekers-Rousseeuw (z-score norm.)  & & 2.6 & 3.3 & 3.7 &  9.8 &  8.5 &  5.6 &  24.1 & 11.1 &  15.9 \\
Raymaekers-Rousseeuw (robust scaling) & & 2.6 & 3.3 & 3.7 &  7.4 &  8.6 &  5.6 &  15.4 & 11.1 &  15.1 \\
invariant                             & & 3.0 & 3.5 & 3.6 &  5.0 &  8.1 &  5.6 &  13.1 & 10.9 &  17.8 \\
robust invariant                      & & 3.0 & 3.4 & 3.6 &  5.5 & 10.1 &  6.2 &  18.7 & 13.2 &  32.8 \\

\bottomrule
\end{tabular}
}
\end{center}
\end{table}

\begin{table}
\begin{center}
\caption{Transformation parameter $\lambda$ for features from simulated clean data without outliers after Yeo-Johnson transformation to normality.
Several transformation methods include normalisation before transformation, indicated by z-score normalisation (norm.) or robust scaling.}
\label{tab:clean-transformation-appendix-lambda-yj}
\small{
\begin{tabular}{l | l r r r r r r r r r}

\toprule
& n: & \multicolumn{3}{c}{30} & \multicolumn{3}{c}{100} & \multicolumn{3}{c}{500} \\
transformation & $\lambda$: & 0.1 & 1.0 & 1.9 & 0.1 & 1.0 & 1.9 & 0.1 & 1.0 & 1.9 \\

\midrule

conventional                          & & 0.6 & 1.2 & 1.5 &  0.0 & 0.9 & 1.6 &  0.1 & 1.0 & 1.9 \\
conventional (z-score norm.)          & & 0.5 & 1.2 & 1.5 & -0.2 & 0.9 & 1.7 & -0.2 & 1.0 & 2.2 \\
conventional (robust scaling)         & & 0.5 & 1.3 & 1.6 & -0.3 & 0.9 & 1.7 & -0.2 & 1.0 & 2.1 \\
Raymaekers-Rousseeuw                  & & 1.0 & 1.2 & 1.5 &  0.0 & 0.7 & 1.6 &  0.1 & 1.0 & 1.9 \\
Raymaekers-Rousseeuw (z-score norm.)  & & 0.7 & 1.2 & 1.5 & -0.2 & 0.7 & 1.7 & -0.2 & 1.0 & 2.2 \\
Raymaekers-Rousseeuw (robust scaling) & & 0.6 & 1.3 & 1.6 & -0.3 & 0.6 & 1.7 & -0.2 & 1.0 & 2.1 \\
invariant                             & & 0.4 & 1.4 & 1.7 &  0.1 & 0.9 & 1.8 &  0.1 & 1.1 & 3.0 \\
robust invariant                      & & 0.5 & 1.1 & 1.5 & -0.1 & 0.7 & 1.7 &  0.1 & 1.1 & 1.9 \\
\bottomrule
\end{tabular}
}
\end{center}
\end{table}

\begin{table}
\begin{center}
\caption{
p-values of empirical central normality tests for features from simulated clean data without outliers after Yeo-Johnson transformation to normality.
Several transformation methods include normalisation before transformation, indicated by z-score normalisation (norm.) or robust scaling.}
\label{tab:clean-transformation-appendix-p-value-yj}
\small{
\begin{tabular}{p{3.5cm} | l r r r r r r r r r}

\toprule
& n: & \multicolumn{3}{c}{30} & \multicolumn{3}{c}{100} & \multicolumn{3}{c}{500} \\
transformation & $\lambda$: & 0.1 & 1.0 & 1.9 & 0.1 & 1.0 & 1.9 & 0.1 & 1.0 & 1.9 \\

\midrule

none                                  & & 0.981 & 0.816 & 0.576 & $<0.001$ & 0.099 & 0.011 & $<0.001$ & 0.993 & $<0.001$ \\
conventional                          & & 0.979 & 0.614 & 0.617 &   0.722  & 0.221 & 0.782 &   0.946  & 0.997 &   0.637 \\
conventional (z-score norm.)          & & 0.970 & 0.594 & 0.645 &   0.251  & 0.224 & 0.895 &   0.529  & 0.998 &   0.828 \\
conventional (robust scaling)         & & 0.974 & 0.593 & 0.640 &   0.538  & 0.207 & 0.837 &   0.933  & 0.998 &   0.810 \\
Raymaekers-Rousseeuw                  & & 0.983 & 0.614 & 0.617 &   0.722  & 0.622 & 0.782 &   0.927  & 0.985 &   0.831 \\
Raymaekers-Rousseeuw (z-score norm.)  & & 0.985 & 0.594 & 0.645 &   0.251  & 0.623 & 0.895 &   0.564  & 0.984 &   0.807 \\
Raymaekers-Rousseeuw (robust scaling) & & 0.987 & 0.593 & 0.640 &   0.538  & 0.599 & 0.837 &   0.935  & 0.984 &   0.875 \\
invariant                             & & 0.987 & 0.487 & 0.573 &   0.778  & 0.241 & 0.683 &   0.963  & 0.997 &   0.865 \\
robust invariant                      & & 0.992 & 0.753 & 0.667 &   0.909  & 0.739 & 0.628 &   0.968  & 0.994 &   0.680 \\

\bottomrule
\end{tabular}
}
\end{center}
\end{table}

\begin{table}
\begin{center}
\caption{Residual errors for features from simulated clean data without outliers after Box-Cox transformation to normality.
Several transformation methods include normalisation before transformation, indicated by z-score normalisation (norm.) or robust scaling.}
\label{tab:clean-transformation-appendix-residuals-bc}
\small{
\begin{tabular}{l | l r r r r r r r r r}

\toprule
& n: & \multicolumn{3}{c}{30} & \multicolumn{3}{c}{100} & \multicolumn{3}{c}{500} \\
transformation & $\lambda$: & 0.1 & 1.0 & 1.9 & 0.1 & 1.0 & 1.9 & 0.1 & 1.0 & 1.9 \\

\midrule

none                                  & & 5.3 & 3.4 & 3.8 & 45.7 &  8.9 & 6.0 & 221.8 & 10.1 & 29.7 \\
conventional                          & & 2.4 & 3.2 & 3.5 &  5.9 &  8.4 & 5.5 &  14.3 &  9.8 & 16.1 \\
conventional (z-score norm.)          & & 2.4 & 3.2 & 3.7 &  5.4 &  8.7 & 5.5 &  14.0 & 10.4 & 16.1 \\
conventional (robust scaling)         & & 2.4 & 3.2 & 3.7 &  5.4 &  8.7 & 5.5 &  14.0 & 10.4 & 16.1 \\
Raymaekers-Rousseeuw                  & & 2.4 & 3.2 & 3.5 &  5.9 & 13.3 & 5.7 &  16.1 & 13.6 & 16.9 \\
Raymaekers-Rousseeuw (z-score norm.)  & & 2.4 & 3.2 & 4.9 &  6.3 &  8.9 & 6.2 &  19.1 & 19.0 & 19.9 \\
Raymaekers-Rousseeuw (robust scaling) & & 2.4 & 3.2 & 4.9 &  6.3 &  8.9 & 6.2 &  18.4 & 19.0 & 19.9 \\
invariant                             & & 2.4 & 3.2 & 3.7 &  5.4 &  8.7 & 5.5 &  13.7 &  9.9 & 16.0 \\
robust invariant                      & & 2.4 & 4.3 & 4.7 &  8.9 & 20.4 & 6.7 &  14.1 & 15.8 & 37.6 \\

\bottomrule
\end{tabular}
}
\end{center}
\end{table}

\begin{table}
\begin{center}
\caption{Transformation parameter $\lambda$ for features from simulated clean data without outliers after Box-Cox transformation to normality.
Several transformation methods include normalisation before transformation, indicated by z-score normalisation (norm.) or robust scaling.}
\label{tab:clean-transformation-appendix-lambda-bc}
\small{
\begin{tabular}{l | l r r r r r r r r r}

\toprule
& n: & \multicolumn{3}{c}{30} & \multicolumn{3}{c}{100} & \multicolumn{3}{c}{500} \\
transformation & $\lambda$: & 0.1 & 1.0 & 1.9 & 0.1 & 1.0 & 1.9 & 0.1 & 1.0 & 1.9 \\

\midrule

conventional                          & & 0.4 & 1.3 & 0.5 & 0.0 &  0.9 & 0.7 & 0.1 & 1.1 & 2.0 \\
conventional (z-score norm.)          & & 0.4 & 1.1 & 0.8 & 0.2 &  0.9 & 0.8 & 0.1 & 1.0 & 1.4 \\
conventional (robust scaling)         & & 0.4 & 1.1 & 0.8 & 0.2 &  0.9 & 0.8 & 0.1 & 1.0 & 1.4 \\
Raymaekers-Rousseeuw                  & & 0.4 & 1.3 & 0.5 & 0.0 & -0.1 & 0.5 & 0.1 & 0.9 & 1.8 \\
Raymaekers-Rousseeuw (z-score norm.)  & & 0.4 & 1.1 & 0.4 & 0.1 &  0.8 & 0.7 & 0.1 & 0.9 & 1.2 \\
Raymaekers-Rousseeuw (robust scaling) & & 0.4 & 1.1 & 0.4 & 0.1 &  0.8 & 0.7 & 0.1 & 0.9 & 1.2 \\
invariant                             & & 0.4 & 1.6 & 0.8 & 0.2 &  0.9 & 0.8 & 0.1 & 1.1 & 1.9 \\
robust invariant                      & & 0.5 & 0.8 & 0.4 & 0.0 &  0.2 & 0.7 & 0.1 & 0.9 & 0.9 \\

\bottomrule
\end{tabular}
}
\end{center}
\end{table}

\begin{table}
\begin{center}
\caption{
p-values of empirical central normality tests for features from simulated clean data without outliers after Box-Cox transformation to normality.
Several transformation methods include normalisation before transformation, indicated by z-score normalisation (norm.) or robust scaling.}
\label{tab:clean-transformation-appendix-p-value-bc}
\small{
\begin{tabular}{p{3.5cm} | l r r r r r r r r r}

\toprule
& n: & \multicolumn{3}{c}{30} & \multicolumn{3}{c}{100} & \multicolumn{3}{c}{500} \\
transformation & $\lambda$: & 0.1 & 1.0 & 1.9 & 0.1 & 1.0 & 1.9 & 0.1 & 1.0 & 1.9 \\

\midrule

none                                  & & 0.940 & 0.816 & 0.498 & $<0.001$ & 0.098 & 0.714 & $<0.001$ & 0.993 & 0.928 \\
conventional                          & & 0.948 & 0.695 & 0.603 &   0.798  & 0.163 & 0.791 &   0.931  & 0.997 & 0.658 \\
conventional (z-score norm.)          & & 0.947 & 0.777 & 0.624 &   0.611  & 0.182 & 0.788 &   0.969  & 0.991 & 0.822 \\
conventional (robust scaling)         & & 0.947 & 0.777 & 0.624 &   0.611  & 0.182 & 0.788 &   0.969  & 0.991 & 0.822 \\
Raymaekers-Rousseeuw                  & & 0.948 & 0.695 & 0.603 &   0.798  & 0.768 & 0.806 &   0.816  & 0.962 & 0.782 \\
Raymaekers-Rousseeuw (z-score norm.)  & & 0.947 & 0.777 & 0.753 &   0.913  & 0.255 & 0.784 &   0.775  & 0.866 & 0.921 \\
Raymaekers-Rousseeuw (robust scaling) & & 0.947 & 0.777 & 0.753 &   0.913  & 0.255 & 0.784 &   0.807  & 0.866 & 0.921 \\
invariant                             & & 0.947 & 0.661 & 0.624 &   0.611  & 0.182 & 0.789 &   0.956  & 0.997 & 0.664 \\
robust invariant                      & & 0.969 & 0.824 & 0.750 &   0.972  & 0.779 & 0.775 &   0.956  & 0.934 & 0.840 \\

\bottomrule
\end{tabular}
}
\end{center}
\end{table}

\section{Appendix F: Experimental
results}\label{appendix-f-experimental-results}

The effect of using location- and scale-invariant transformations was
investigated using real-world datasets, as described in the main
manuscript.

\subsection{Yeo-Johnson
transformation}\label{yeo-johnson-transformation}

Additional results for Yeo-Johnson transformations are shown in Tables
\ref{tab:experimental-results-appendix-residuals},
\ref{tab:experimental-results-appendix-lambda},
\ref{tab:experimental-results-appendix-p-value}.

\begin{table}
\begin{center}
\caption{Residual errors for features from real-world datasets after Yeo-Johnson transformation to normality. 
Several transformation methods include normalisation before transformation, indicated by z-score normalisation (norm.) or robust scaling. 
AWT: arterial wall thickness; FE: fuel efficiency; PBM: penguin body mass}
\label{tab:experimental-results-appendix-residuals}
\begin{tabular}{l | r r r r r}

\toprule
feature & age & AWT & FE & latitude & PBM \\

\midrule
none                                  & 16.5 & 110.1 & 54.5 & 328.4 & 48.0 \\
conventional                          & 11.5 &  30.0 & 55.3 & 319.0 & 32.2 \\
conventional (z-score norm.)          & 11.5 &  19.3 & 49.0 & 326.2 & 33.3 \\
conventional (robust scaling)         & 11.3 &  31.8 & 53.3 & 324.5 & 32.2 \\
Raymaekers-Rousseeuw                  & 11.5 & 136.7 & 47.7 & 315.1 & 32.2 \\
Raymaekers-Rousseeuw (z-score norm.)  & 13.2 & 214.5 & 57.4 & 326.2 & 33.3 \\
Raymaekers-Rousseeuw (robust scaling) & 11.7 & 281.3 & 57.0 & 324.5 & 32.0 \\
invariant                             &  8.8 &  12.2 & 44.0 & 326.4 & 26.8 \\
robust invariant                      &  9.3 &  30.2 & 55.8 & 308.1 & 22.0 \\

\bottomrule
\end{tabular}
\end{center}
\end{table}

\begin{table}
\begin{center}
\caption{Transformation parameter $\lambda$ for features from real-world datasets after Yeo-Johnson transformation to normality. 
Several transformation methods include normalisation before transformation, indicated by z-score normalisation (norm.) or robust scaling. 
AWT: arterial wall thickness; FE: fuel efficiency; PBM: penguin body mass}
\label{tab:experimental-results-appendix-lambda}
\begin{tabular}{l | r r r r r}

\toprule
feature & age & AWT & FE & latitude & PBM \\

\midrule
conventional                          & 2.0 & -0.7 & -0.1 & 62.1 & -0.5 \\
conventional (z-score norm.)          & 1.2 & -1.7 & -0.1 &  1.3 &  0.6 \\
conventional (robust scaling)         & 1.3 &  0.0 &  0.2 &  1.4 &  0.4 \\
Raymaekers-Rousseeuw                  & 2.0 &  1.1 &  0.8 & 95.4 & -0.5 \\
Raymaekers-Rousseeuw (z-score norm.)  & 1.3 &  1.4 &  1.0 &  1.3 &  0.6 \\
Raymaekers-Rousseeuw (robust scaling) & 1.4 &  1.3 &  1.0 &  1.4 &  0.4 \\
invariant                             & 1.3 &  0.2 & -1.3 &  1.5 &  0.5 \\
robust invariant                      & 1.3 & -0.3 &  1.0 &  1.1 &  0.3 \\
\bottomrule
\end{tabular}
\end{center}
\end{table}

\begin{table}
\begin{center}
\caption{
p-values of empirical central normality tests for features from real-world datasets after Yeo-Johnson transformation to normality.
Several transformation methods include normalisation before transformation, indicated by z-score normalisation (norm.) or robust scaling. 
AWT: arterial wall thickness; FE: fuel efficiency; PBM: penguin body mass}
\label{tab:experimental-results-appendix-p-value}
\begin{tabular}{l | r r r r r}

\toprule
feature & age & AWT & FE & latitude & PBM \\

\midrule

none                                  & 0.694 & 0.832     & 0.727     & $< 0.001$ & $< 0.001$ \\
conventional                          & 0.961 & 0.003     & $< 0.001$ & $< 0.001$ & 0.101 \\
conventional (z-score norm.)          & 0.919 & 0.021     & $< 0.001$ & $< 0.001$ & 0.099 \\
conventional (robust scaling)         & 0.933 & $< 0.001$ & $< 0.001$ & $< 0.001$ & 0.129 \\
Raymaekers-Rousseeuw                  & 0.961 & 0.881     & 0.373     & $< 0.001$ & 0.101 \\
Raymaekers-Rousseeuw (z-score norm.)  & 0.694 & 0.899     & 0.809     & $< 0.001$ & 0.099 \\
Raymaekers-Rousseeuw (robust scaling) & 0.900 & 0.880     & 0.811     & $< 0.001$ & 0.144 \\
invariant                             & 0.976 & 0.147     & $< 0.001$ & $< 0.001$ & 0.276 \\
robust invariant                      & 0.933 & 0.178     & 0.759     & $< 0.001$ & 0.688 \\

\bottomrule
\end{tabular}
\end{center}
\end{table}

\subsection{Box-Cox transformation}\label{box-cox-transformation}

Results for Box-Cox transformations are shown in Tables
\ref{tab:experimental-results-appendix-residuals-bc},
\ref{tab:experimental-results-appendix-lambda-bc} and
\ref{tab:experimental-results-appendix-p-value-bc}.

\begin{table}
\begin{center}
\caption{Residual errors for features from real-world datasets after Box-Cox transformation to normality. 
Several transformation methods include normalisation before transformation, indicated by z-score normalisation (norm.) or robust scaling. 
AWT: arterial wall thickness; FE: fuel efficiency; PBM: penguin body mass}
\label{tab:experimental-results-appendix-residuals-bc}
\begin{tabular}{l | r r r r r}

\toprule
feature & age & AWT & FE & latitude & PBM \\

\midrule
none                                  & 16.5 & 110.1 &  54.5 & 328.4 & 48.0 \\
conventional                          & 11.5 &  33.5 &  55.7 & 318.9 & 32.2 \\
conventional (z-score norm.)          & 12.9 &  44.5 &  58.8 & 305.3 & 27.3 \\
conventional (robust scaling)         & 12.9 &  44.7 &  59.0 & 305.3 & 27.3 \\
Raymaekers-Rousseeuw                  & 11.5 & 127.1 &  47.6 & 314.8 & 32.2 \\ 
Raymaekers-Rousseeuw (z-score norm.)  & 13.1 &  54.8 & 127.5 & 510.2 & 23.6 \\
Raymaekers-Rousseeuw (robust scaling) & 18.9 & 100.2 & 113.7 & 510.2 & 23.6 \\
invariant                             & 11.6 &  28.0 &  48.4 & 311.8 & 27.3 \\
robust invariant                      & 12.8 & 150.1 &  60.5 & 646.0 & 27.4 \\

\bottomrule
\end{tabular}
\end{center}
\end{table}

\begin{table}
\begin{center}
\caption{Transformation parameter $\lambda$ for features from real-world datasets after Box-Cox transformation to normality. 
Several transformation methods include normalisation before transformation, indicated by z-score normalisation (norm.) or robust scaling. 
AWT: arterial wall thickness; FE: fuel efficiency; PBM: penguin body mass}
\label{tab:experimental-results-appendix-lambda-bc}
\begin{tabular}{l | r r r r r}

\toprule
feature & age & AWT & FE & latitude & PBM \\

\midrule

conventional                          & 1.9 & -0.5 & -0.1 & 62.1 & -0.5 \\
conventional (z-score norm.)          & 1.2 &  0.1 &  0.2 &  1.2 &  0.5 \\
conventional (robust scaling)         & 1.2 &  0.1 &  0.1 &  1.2 &  0.5 \\
Raymaekers-Rousseeuw                  & 1.9 &  1.1 &  0.8 & 95.9 & -0.5 \\
Raymaekers-Rousseeuw (z-score norm.)  & 1.2 &  0.6 & -0.5 &  0.6 &  0.3 \\
Raymaekers-Rousseeuw (robust scaling) & 0.9 &  1.0 & -0.5 &  0.6 &  0.3 \\
invariant                             & 1.7 & -1.0 & -0.7 &  1.9 &  0.5 \\
robust invariant                      & 1.3 &  1.2 &  1.1 &  0.4 &  0.1 \\

\bottomrule
\end{tabular}
\end{center}
\end{table}

\begin{table}
\begin{center}
\caption{
p-values of empirical central normality tests for features from real-world datasets after Box-Cox transformation to normality.
Several transformation methods include normalisation before transformation, indicated by z-score normalisation (norm.) or robust scaling. 
AWT: arterial wall thickness; FE: fuel efficiency; PBM: penguin body mass}
\label{tab:experimental-results-appendix-p-value-bc}
\begin{tabular}{l | r r r r r}

\toprule
feature & age & AWT & FE & latitude & PBM \\

\midrule

none                                  & 0.694 & 0.832 & 0.727     & $< 0.001$ & $< 0.001$ \\
conventional                          & 0.961 & 0.002 & $< 0.001$ & $< 0.001$ & 0.101 \\
conventional (z-score norm.)          & 0.901 & 0.005 & $< 0.001$ & $< 0.001$ & 0.154 \\
conventional (robust scaling)         & 0.901 & 0.005 & $< 0.001$ & $< 0.001$ & 0.154 \\
Raymaekers-Rousseeuw                  & 0.961 & 0.871 & 0.365     & $< 0.001$ & 0.101 \\
Raymaekers-Rousseeuw (z-score norm.)  & 0.891 & 0.268 & $< 0.001$ & $< 0.001$ & 0.495 \\
Raymaekers-Rousseeuw (robust scaling) & 0.562 & 0.789 & $< 0.001$ & $< 0.001$ & 0.492 \\
invariant                             & 0.962 & 0.004 & $< 0.001$ & $< 0.001$ & 0.154 \\
robust invariant                      & 0.899 & 0.890 & 0.866     & $< 0.001$ & 0.622 \\

\bottomrule
\end{tabular}
\end{center}
\end{table}

\section*{References}\label{references}
\addcontentsline{toc}{section}{References}

\phantomsection\label{refs}
\begin{CSLReferences}{1}{0}
\bibitem[\citeproctext]{ref-Anderson1952-gz}
Anderson, T W, and D A Darling. 1952. {``Asymptotic Theory of Certain
{`Goodness of Fit'} Criteria Based on Stochastic Processes.''}
\emph{Annals of Mathematical Statistics} 23 (2): 193--212.
\url{https://doi.org/10.1214/aoms/1177729437}.

\bibitem[\citeproctext]{ref-Anscombe1983-nz}
Anscombe, F J, and William J Glynn. 1983. {``Distribution of the
Kurtosis Statistic B2 for Normal Samples.''} \emph{Biometrika} 70 (1):
227--34. \url{https://doi.org/10.1093/biomet/70.1.227}.

\bibitem[\citeproctext]{ref-Cramer1928-rc}
Cramér, Harald. 1928. {``On the Composition of Elementary Errors.''}
\emph{Scand. Actuar. J.} 1928 (1): 13--74.
\url{https://doi.org/10.1080/03461238.1928.10416862}.

\bibitem[\citeproctext]{ref-DAgostino1990-kp}
D'Agostino, Ralph B, and Albert Belanger. 1990. {``A Suggestion for
Using Powerful and Informative Tests of Normality.''} \emph{Am. Stat.}
44 (4): 316--21. \url{https://doi.org/10.2307/2684359}.

\bibitem[\citeproctext]{ref-Huber1981-su}
Huber, Peter J. 1981. \emph{Robust Statistics}. John Wiley \& Sons.
\url{https://doi.org/10.1002/0471725250}.

\bibitem[\citeproctext]{ref-Jarque1980-hw}
Jarque, Carlos M, and Anil K Bera. 1980. {``Efficient Tests for
Normality, Homoscedasticity and Serial Independence of Regression
Residuals.''} \emph{Econ. Lett.} 6 (3): 255--59.
\url{https://doi.org/10.1016/0165-1765(80)90024-5}.

\bibitem[\citeproctext]{ref-Von_Mises1928-ef}
Mises, Richard von. 1928. \emph{Wahrscheinlichkeit Statistik Und
Wahrheit}. Schriften Zur Wissenschaftlichen Weltauffassung.
Springer-Verlag Berlin, Heidelberg.
\url{https://doi.org/10.1007/978-3-662-36230-3}.

\bibitem[\citeproctext]{ref-Raymaekers2024-zf}
Raymaekers, Jakob, and Peter J Rousseeuw. 2024. {``Transforming
Variables to Central Normality.''} \emph{Mach. Learn.} 113 (8):
4953--75. \url{https://doi.org/10.1007/s10994-021-05960-5}.

\bibitem[\citeproctext]{ref-Shapiro1965-zd}
Shapiro, S S, and M B Wilk. 1965. {``An Analysis of Variance Test for
Normality (Complete Samples).''} \emph{Biometrika} 52 (3/4): 591--611.
\url{https://doi.org/10.2307/2333709}.

\end{CSLReferences}

\end{document}
